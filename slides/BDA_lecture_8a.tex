\documentclass[t]{beamer}
%\documentclass[finnish,english,handout]{beamer}

% Uncomment if want to show notes
% \setbeameroption{show notes}

\mode<presentation>
{
  % \usetheme{Warsaw}
}
\setbeamertemplate{itemize items}[circle]
\setbeamercolor{frametitle}{bg=white,fg=navyblue}


% \usepackage[pdftex]{graphicx}
%\usepackage[T1]{fontenc}
\usepackage[utf8]{inputenc}
\usepackage{times}
\usepackage{microtype}
\usepackage{epic,epsfig}
%\usepackage{subfigure,float}
\usepackage{amsmath,amsfonts,amssymb}
\usepackage{inputenc}
\usepackage{afterpage}
\usepackage{url}
\urlstyle{same}
\usepackage{amsbsy}
\usepackage{eucal}
\usepackage{rotating}
\usepackage{subcaption}
\usepackage{listings}
\usepackage{lstbayes}
\usepackage[all,poly,ps,color]{xy}
\usepackage{eurosym}

\usepackage{natbib}
\bibliographystyle{apalike}

% \definecolor{hutblue}{rgb}{0,0.2549,0.6784}
% \definecolor{midnightblue}{rgb}{0.0977,0.0977,0.4375}
% \definecolor{hutsilver}{rgb}{0.4863,0.4784,0.4784}
% \definecolor{lightgray}{rgb}{0.95,0.95,0.95}
% \definecolor{section}{rgb}{0,0.2549,0.6784}
% \definecolor{list1}{rgb}{0,0.2549,0.6784}
\definecolor{forestgreen}{rgb}{0.1333,0.5451,0.1333}
\definecolor{navyblue}{rgb}{0,0,0.5}
\renewcommand{\emph}[1]{\textcolor{navyblue}{#1}}
\definecolor{darkgreen}{rgb}{0,0.3922,0}

%%%%%%%%%%%%%%%%%%% for hiding figures %%%%%%%%%%%%%%%%%%%%%%%%%%
\usepackage{color}
\newcommand{\hide}[5][white]{
	% usage: \hhide[color]{vspace,hspace,height,width}
	% note: all measures are relative units measured in \textwidth
	%\begin{minipage}{0.99\textwidth}
	\vspace{#2\textwidth}
	\hspace{#3\textwidth}
	\textcolor{#1}{  \rule{#5\textwidth}{#4\textwidth}  }
	% \end{minipage}
      }

\graphicspath{{./figs/}}

\pdfinfo{
  /Title      (Bayesian data analysis, ch 6)
  /Author     (Aki Vehtari) %
  /Keywords   (Bayesian probability theory, Bayesian inference, Bayesian data analysis)
}


\parindent=0pt
\parskip=8pt
\tolerance=9000
\abovedisplayshortskip=2pt

\setbeamertemplate{navigation symbols}{}
\setbeamertemplate{headline}[default]{}
\setbeamertemplate{headline}[text line]{\insertsection}
\setbeamertemplate{footline}[frame number]


\def\o{{\mathbf o}}
\def\t{{\mathbf \theta}}
\def\w{{\mathbf w}}
\def\x{{\mathbf x}}
\def\y{{\mathbf y}}
\def\z{{\mathbf z}}

\def\peff{p_{\mathrm{eff}}}
\def\eff{\mathrm{eff}}

\DeclareMathOperator{\E}{E}
\DeclareMathOperator{\Var}{Var}
\DeclareMathOperator{\var}{var}
\DeclareMathOperator{\Sd}{Sd}
\DeclareMathOperator{\sd}{sd}
\DeclareMathOperator{\Gammad}{Gamma}
\DeclareMathOperator{\Invgamma}{Inv-gamma}
\DeclareMathOperator{\Bin}{Bin}
\DeclareMathOperator{\Negbin}{Neg-bin}
\DeclareMathOperator{\Poisson}{Poisson}
\DeclareMathOperator{\Beta}{Beta}
\DeclareMathOperator{\logit}{logit}
\DeclareMathOperator{\N}{N}
\DeclareMathOperator{\normal}{normal}
\DeclareMathOperator{\U}{U}
\DeclareMathOperator{\BF}{BF}
\DeclareMathOperator{\Invchi2}{Inv-\chi^2}
\DeclareMathOperator{\NInvchi2}{N-Inv-\chi^2}
\DeclareMathOperator{\InvWishart}{Inv-Wishart}
\DeclareMathOperator{\tr}{tr}
% \DeclareMathOperator{\Pr}{Pr}
\def\euro{{\footnotesize \EUR\, }}
\DeclareMathOperator{\rep}{\mathrm{rep}}

% \def\dashxy(#1){%
%   /xydash{[#1] 0 setdash}def}
% \def\grayxy(#1){%
%   /xycolor{#1 setgray}def}
% \newgraphescape{D}[1]{!{\ar @*{[!\dashxy(2 2)]} "#1"}}
% \newgraphescape{P}[1]{!{\ar "#1"}}
% \newgraphescape{F}[1]{!{*+=<2em>[F=]{#1}="#1"}}
% \newgraphescape{O}[1]{!{*+=<2em>[F]{#1}="#1"}}
% \newgraphescape{V}[1]{!{*+=<2em>[o][F]{#1}="#1"}}
% \newgraphescape{B}[3]{!{{ "#1"*+#3\frm{} }.{ "#2"*+#3\frm{} } *+[F:!\grayxy(0.75)]\frm{}}}


\title[]{Bayesian data analysis}
\subtitle{}

\author{Aki Vehtari}

\institute[Aalto]{}

\begin{document}

\begin{frame}{Chapter 6}

  \begin{itemize}
  \item 6.1 The place of model checking in applied Bayesian statistics
  \item 6.2 Do the inferences from the model make sense?
  \item 6.3 Posterior predictive checking
  \item {\color{gray} 6.4 Graphical posterior predictive checks}    
    \begin{itemize}
    \item[-] this can be skimmed, see instead the paper\\ Gabry et al. (2019). \href{https://doi.org/10.1111/rssa.12378}{\textit{Visualization in Bayesian workflow}}
  \end{itemize}
\item 6.5 Model checking for the educational testing example
  \end{itemize}
  
\end{frame}

\begin{frame}
  
  {\Large\color{navyblue} Model checking}

  \begin{itemize}
  \item demo6\_1: Posterior predictive checking - light speed
  \item demo6\_2: Posterior predictive checking - sequential dependence
  \item demo6\_3: Posterior predictive checking - poor test statistic
  \item demo6\_4: Posterior predictive checking - marginal predictive p-value
  \end{itemize}

\end{frame}

 \begin{frame}{Model checking -- overview}

  \begin{itemize}
  \item<+-> Sensibility with respect to additional information not used in modeling
    \begin{itemize}
    \item e.g., if posterior would claim that hazardous chemical
      decreases probability of death
    \end{itemize}
  \item<+-> External validation
    \begin{itemize}
    \item compare predictions to completely new observations
    \item cf. relativity theory predictions
    \end{itemize}
  \item<+-> Internal validation
    \begin{itemize}
    \item posterior predictive checking
    \item cross-validation predictive checking
    \end{itemize}
  \end{itemize}

\end{frame}

\begin{frame}{Posterior predictive checking -- example}

  \begin{itemize}
  \item<1-> Newcomb's speed of light measurements
    \begin{itemize}
    \item model $y\sim\normal(\mu,\sigma)$ with prior $(\mu,\log\sigma)\propto 1$
    \end{itemize}
  \item<2-> Posterior predictive replicate $y^{\rm rep}$
    \begin{itemize}
    \item<3-> draw $\mu^{(s)},\sigma^{(s)}$ from the posterior $p(\mu,\sigma|y)$
    \item<4-> draw $y^{\mathrm{rep}\,(s)}$ from $\normal(\mu^{(s)},\sigma^{(s)})$
    \item<5-> repeat $n$ times to get $y^{\mathrm{rep}}$ with $n$ replicates\\~\\
    \uncover<6->{\includegraphics[width=7cm]{light_ppc_1hist.pdf}}
      \end{itemize}
    \end{itemize}

\end{frame}

\begin{frame}{Replicates vs. future observation}

  \begin{itemize}
  \item Predictive $\tilde{y}$ is the next not yet observed possible
    observation. $y^{\mathrm{rep}}$ refers to replicating the whole
    experiment (potentially with same values of $x$) and obtaining as
    many replicated observations as in the original data.
  \end{itemize}

\end{frame}

\begin{frame}{Posterior predictive checking -- example}

  \begin{itemize}
  \item<1-> Generate several replicated datasets $y^{\rm rep}$
  \item<2-> Compare to the original dataset
  \end{itemize}
  \vspace{-1\baselineskip}
  \uncover<3->{\includegraphics[width=11.5cm]{light_ppc_10hist.pdf}}

\end{frame}

\begin{frame}{Posterior predictive checking -- bayesplot}

  \vspace{-1\baselineskip}
  \texttt{ppc\_hist(y, yrep[1:8,])}
  
  \includegraphics[height=8cm]{Newcomb_ppc_hist.pdf}

\end{frame}

\begin{frame}{Posterior predictive checking -- bayesplot}

  \vspace{-1\baselineskip}
  \texttt{ppc\_dens\_overlay(y, yrep[1:100,])}
  
  \includegraphics[height=8cm]{Newcomb_ppc_dens_overlay.pdf}

\end{frame}

\begin{frame}{Posterior predictive checking -- bayesplot}

  \vspace{-1\baselineskip}
  \texttt{ppc\_ecdf\_overlay(y, yrep[1:100,])}
  
  \includegraphics[height=8cm]{Newcomb_ppc_ecdf_overlay.pdf}

\end{frame}

\begin{frame}{Posterior predictive checking -- bayesplot}

  \vspace{-1\baselineskip}
  \texttt{ppc\_scatter(y, yrep[1:4,]) + geom\_abline()}
  
  \includegraphics[height=7.5cm]{Newcomb_ppc_scatter.pdf}

\end{frame}

\begin{frame}{Posterior predictive checking with test statistic}

  \begin{itemize}
  \item Replicated data sets $y^{\rep}$
  \item Test quantity (or discrepancy measure) $T(y,\theta)$
    \begin{itemize}
    \item summary quantity for the observed data $T(y,\theta)$
    \item summary quantity for a replicated data $T(y^{\rep},\theta)$
    \item can be easier to compare summary quantities than data sets
    \end{itemize}
  \end{itemize}

\end{frame}

\begin{frame}{Posterior predictive checking -- example}

  \begin{itemize}
  \item<1-> Compute test statistic for data $T(y,\theta)=\min(y)$
  \item<2-> Compute test statistic $\min(y^{\rm rep})$ for many replicated datasets 
  \end{itemize}
  \vspace{-1.5\baselineskip}
  \uncover<3->{\includegraphics[width=11cm]{light_ppc_min.pdf}}

\end{frame}

\begin{frame}{Posterior predictive checking -- example}

  \begin{itemize}
  \item<1-> Good test statistic is ancillary (or almost)
    \begin{itemize}
    \item ancillary if it depends only on observed data and if its
      distribution is independent of the parameters of the model
    \end{itemize}
  \item<2-> Bad test statistic is highly dependent of the parameters
    \begin{itemize}
    \item e.g. variance for normal model
    \end{itemize}
  \end{itemize}
  \vspace{-1.5\baselineskip}
  \uncover<3->{\includegraphics[width=10cm]{light_ppc_var.pdf}}

\end{frame}

\begin{frame}{Posterior predictive checking -- bayesplot}

  \vspace{-1\baselineskip}
  \texttt{ppc\_stat(y, yrep)}, the default statistic "mean" is usually bad
  
  \includegraphics[height=8cm]{Newcomb_ppc_stat_mean.pdf}

\end{frame}

\begin{frame}{Posterior predictive checking -- bayesplot}

  \vspace{-1\baselineskip}
  \texttt{ppc\_stat(y, yrep, stat="min")}
  
  \includegraphics[height=8cm]{Newcomb_ppc_stat_min.pdf}

\end{frame}

\begin{frame}{Posterior predictive checking -- bayesplot}

  \vspace{-1\baselineskip}
  \texttt{ppc\_stat(y, yrep, stat="max")}
  
  \includegraphics[height=8cm]{Newcomb_ppc_stat_max.pdf}

\end{frame}

\begin{frame}{Posterior predictive checking -- bayesplot}

  \vspace{-1\baselineskip}
  \texttt{ppc\_stat2d(y, yrep, stat=c("min","max"))}
  
  \includegraphics[height=8cm]{Newcomb_ppc_stat2d_minmax.pdf}

\end{frame}

\begin{frame}{Posterior predictive checking}

  \begin{itemize}
    \only<4->{\color{gray}}
  \item<1-> \textit{Posterior predictive $p$-value}
    \begin{eqnarray*}
      p & = & \Pr(T(y^{\rep},\theta)\geq T(y,\theta)|y)\\
      & = & \int\int
      I_{T(y^{\rep},\theta)\geq T(y,\theta)}p(y^{\rep}|\theta)p(\theta|y)dy^{\rep}d\theta
    \end{eqnarray*}
    where $I$ is an indicator function
    \begin{itemize}
    \item<2-> \only<4->{\color{gray}} having $(y^{\rep\,(s)},\theta^{(s)})$ from the posterior predictive
      distribution, easy to compute
      \begin{equation*}
        T(y^{\rep (s)},\theta^{(s)})\geq T(y,\theta^{(s)}), \quad s=1,\ldots,S
      \end{equation*}
    \end{itemize}
    \vspace{-1.5\baselineskip}
  \item<3-> Posterior predictive $p$-value (ppp-value) estimates whether
    difference between the model and data could arise by chance
  \item<4-> \color{black} Not commonly used, as
    \begin{itemize}
    \item the distribution of test statistic has more information
    \item not calibrated in case of non-ancillary statistic
    \end{itemize}
  \end{itemize}

\end{frame}

\begin{frame}{Posterior predictive checking -- example}

  \begin{itemize}
  \item In general it is often better to improve the model than remove
    data
  \item There can be cases where after the first model fit, it is
    clear that some data are clearly incorrectly measured, but there
    has to be really strong justifications for dropping out
    observations
  \item<2-> Let's assume that in Newcomb experiment, the two observations
    with negative values are clearly incorrect and let's refit by
    removing them
  \end{itemize}

\end{frame}

  
\begin{frame}{Posterior predictive checking (good fit)}

  \vspace{-1\baselineskip}
  \texttt{ppc\_hist(y2, y2rep[1:8,])}
  
  \includegraphics[height=8cm]{Newcomb_ppc_hist_nooutliers.pdf}

\end{frame}

\begin{frame}{Posterior predictive checking (good fit)}

  \vspace{-1\baselineskip}
  \texttt{ppc\_dens\_overlay(y2, y2rep[1:100,])}
  
  \includegraphics[height=8cm]{Newcomb_ppc_dens_overlay_nooutliers.pdf}

\end{frame}

\begin{frame}{Posterior predictive checking (good fit)}

  \vspace{-1\baselineskip}
  \texttt{ppc\_ecdf\_overlay(y2, y2rep[1:100,])}
  
  \includegraphics[height=8cm]{Newcomb_ppc_ecdf_overlay_nooutliers.pdf}

\end{frame}

\begin{frame}{Posterior predictive checking (good fit)}

  \vspace{-1\baselineskip}
  \texttt{ppc\_stat2d(y2, y2rep, stat=c("min","max"))}
  
  \includegraphics[height=8cm]{Newcomb_ppc_stat2d_minmax_nooutliers.pdf}

\end{frame}

% \begin{frame}

%   {\Large\color{navyblue} Calibration of ppp-values}

%   \begin{itemize}
%   \item In the special case that the parameters $\theta$ are known (or
%     estimated to a very high precision) or in which the test statistic
%     $T(y)$ is ancillary (that is,
%     if it depends only on observed data and if its distribution is
%     independent of the parameters of the model) with a continuous
%     distribution, the posterior predictive $p$-value
%     $\Pr(T(y^{\rep})\!>\!T(y)|y)$ has a distribution that is uniform
%     if the model is true.
%   \item Under these conditions, $p$-values less than 0.1 occur 10\% of
%     the time, $p$-values less than 0.05 occur 5\% of the time, and so
%     forth.
%   \end{itemize}

% \end{frame}

\begin{frame}{Marginal and CV predictive checking}

  \begin{itemize}
  \item Consider marginal predictive distributions $p(\tilde{y}_i|y)$
    and each observation separately
    \begin{itemize}
    \item marginal posterior p-values
      \begin{align*}
        p_i = \mbox{Pr}(T(y_i^{\rep}) \leq T(y_i) | y)
      \end{align*}
      if $T(y_i)=y_i$
      \begin{align*}
        p_i = \mbox{Pr}(y_i^{\rep} \leq y_i | y)
      \end{align*}
    \end{itemize}
  \item<2-> if $Pr(\tilde{y}_i|y)$ well calibrated, distribution of $p_i$
    would be uniform between 0 and 1
    \begin{itemize}
    \item holds better for cross-validation predictive tests
      (cross-validation BDA3 Ch 7) 
    \end{itemize}
  \end{itemize}

\end{frame}

\begin{frame}{Marginal predictive checking (good fit)}

  \vspace{-0.5\baselineskip}
  \begin{itemize}
  \item Marginal tail area or Probability integral transform (PIT)
    \begin{align*}
      p_i = p(y_i^{\rep} \leq y_i | y)
    \end{align*}
  \item if $p(\tilde{y}_i|y)$ is well calibrated, distribution of $p_i$'s
    would be uniform between 0 and 1
  \end{itemize}
  \vspace{-1\baselineskip}
  \only<2>{\includegraphics[width=9.5cm]{Newcomb_PIT_hist.pdf}}
  \only<3>{\includegraphics[width=9.5cm]{Newcomb_PIT_ecdf.pdf}}
  \only<4>{\includegraphics[width=9.5cm]{Newcomb_PIT_ecdf_nooutliers.pdf}}


\end{frame}

\begin{frame}{Example: Exposure to air pollution}

  \begin{itemize}
  \item Example from Jonah Gabry, Daniel Simpson, Aki Vehtari, Michael
    Betancourt, and Andrew Gelman (2019). Visualization in Bayesian
    workflow. \url{https://doi.org/10.1111/rssa.12378}
  \item Estimation of human exposure to air pollution from particulate
    matter measuring less than 2.5 microns in diameter ($\mathrm{PM}_{2.5}$)
    \begin{itemize}
    \item Exposure to $\mathrm{PM}_{2.5}$ is linked to a number of
      poor health outcomes and a recent report estimated that
      $\mathrm{PM}_{2.5}$ is responsible for three million deaths
      worldwide each year (Shaddick et al., 2017)
    \item In order to estimate the public health effect of ambient
      $\mathrm{PM}_{2.5}$, we need a good estimate of the
      $\mathrm{PM}_{2.5}$ concentration at the same spatial resolution
      as our population estimates.
    \end{itemize}
\end{itemize}

\end{frame}

\begin{frame}{Example: Exposure to air pollution}

  \begin{itemize}
  \item Direct measurements of PM 2.5 from ground monitors at 2980
    locations
  \item High-resolution satellite data of aerosol optical depth
    
  \end{itemize}
  \begin{center}
    \only<1>{\vspace{-1.8\baselineskip}\includegraphics[height=7cm]{map-data.png}}
    \only<2>{\vspace{-1.8\baselineskip}\includegraphics[height=7cm]{plot1.png}}
    \only<3>{\hspace{-3cm}\includegraphics[height=6.55cm]{plot2.png}}
\end{center}
\end{frame}

\begin{frame}{Example: Exposure to air pollution}

  Prior predictive checking
  \vspace{-1\baselineskip}
  \begin{center}
    \only<1>{\includegraphics[width=11cm]{pm25_pp1a.pdf}}
    \only<2>{\includegraphics[width=11cm]{pm25_pp1b.pdf}}
    \only<3>{\includegraphics[width=11cm]{pm25_pp2.pdf}}
\end{center}
\end{frame}

\begin{frame}{Example: Exposure to air pollution}

  Posterior predictive checking -- marginal predictive distributions
\begin{figure}
\centering
\begin{subfigure}{0.48\textwidth}
\includegraphics[width=\textwidth]{ppc_dens1.png}
\caption{Model 1}
\end{subfigure}
~
\begin{subfigure}{0.48\textwidth}
\includegraphics[width=\textwidth]{ppc_dens2.png}
\caption{Model 2}
\end{subfigure}
% ~
% \begin{subfigure}{0.31\textwidth}
% \includegraphics[width=\textwidth]{ppc_dens3.png}
% \caption{Model 3}
% \end{subfigure}
\end{figure}

\end{frame}

\begin{frame}{Example: Exposure to air pollution}


  Posterior predictive checking -- test statistic (skewness)
\begin{figure}
\centering
\begin{subfigure}{0.31\textwidth}
\includegraphics[width=\textwidth]{ppc_skew1.png}
\caption{Model 1}
\end{subfigure}
~
\begin{subfigure}{0.31\textwidth}
\includegraphics[width=\textwidth]{ppc_skew2.png}
\caption{Model 2}
\end{subfigure}
~
\begin{subfigure}{0.31\textwidth}
\includegraphics[width=\textwidth]{ppc_skew3.png}
\caption{Model 3}
\end{subfigure}

\end{figure}

\end{frame}

\begin{frame}{Example: Exposure to air pollution}


  Posterior predictive checking -- test statistic (median for groups)

  \begin{figure}
\centering
\begin{subfigure}{.31\textwidth}
\includegraphics[width=\textwidth]{ppc_med_grouped1.png}
\caption{Model 1}
\end{subfigure}
~
\begin{subfigure}{.31\textwidth}
\includegraphics[width=\textwidth]{ppc_med_grouped2.png}
\caption{Model 2}
\end{subfigure}
~
\begin{subfigure}{.31\textwidth}
\includegraphics[width=\textwidth]{ppc_med_grouped3.png}
\caption{Model 3}
\end{subfigure}

\end{figure}

\end{frame}

\begin{frame}{Example: Exposure to air pollution}


  LOO predictive checking -- LOO-PIT

\includegraphics[width=\textwidth]{ppc_loo_pit_corrected_pm25.png}

% \begin{figure}
% \centering
% \begin{subfigure}{0.31\textwidth}
% \includegraphics[width=\textwidth]{ppc_loo_pit_overlay1.png}
% \caption{Model 1}
% \end{subfigure}
% ~
% \begin{subfigure}{0.31\textwidth}
% \includegraphics[width=\textwidth]{ppc_loo_pit_overlay2.png}
% \caption{Model 2}
% \end{subfigure}
% ~
% \begin{subfigure}{0.31\textwidth}
% \includegraphics[width=\textwidth]{ppc_loo_pit_overlay3.png}
% \caption{Model 3}
% \end{subfigure}
% \end{figure}

% \vspace{2\baselineskip}
% {\scriptsize EDIT 2020: These plots use boundary corrected KDE which is a better
% choice than the non-boundary corrected KDE used in the plots in the
% paper and the 2019 lecture.}

\end{frame}

\begin{frame}{Positive target}
  
    \includegraphics[width=11cm]{mesquite_ppc.pdf}\\
  \vspace{-0.1\baselineskip} {Predicting the yields of mesquite bushes.\\
    \color{gray} \footnotesize
    Gelman, Hill \& Vehtari (2020): Regression and Other Stories, Chapter 11.}\\

\end{frame}

\begin{frame}{PPC for binary target}

  \vspace{-0.75\baselineskip}
  Diabetes prediction with logistic regression -
  \href{https://avehtari.github.io/modelselection/diabetes.html}{diabetes demo}
  
     \only<1>{\vspace{-.6\baselineskip}\includegraphics[width=8.5cm]{diabetes_corrplot.pdf}}
     \only<2>{PPC with binning for binary data\\ \includegraphics[width=11cm]{diabetes_calibration_binned.pdf}}
     % \only<3>{PPC with non-linear regression for binary data\\ \includegraphics[width=11cm]{diabetes_calibration_regression.pdf}}
     \only<3>{PPC with monotonic regression for binary data\\ \includegraphics[width=11cm]{diabetes_reliabilitydiag.pdf}}

\end{frame}

\begin{frame}[fragile]
  \frametitle{Posterior predictive checking}

  \vspace{-0.2\parskip}
  \begin{itemize}
  \item demo demos\_rstan/ppc/poisson-ppc.Rmd
  \end{itemize}

  \vspace{-0.2\parskip}
  {\color{gray}\footnotesize
\begin{lstlisting}[language=Stan]
data {
  int<lower=1> N;
  int<lower=0> y[N];
}
parameters {
  real<lower=0> lambda;
}
model {
  lambda ~ exponential(0.2);
  y ~ poisson(lambda);
}
\end{lstlisting}
  }
  \vspace{-\parskip}
  {\footnotesize
\begin{lstlisting}[language=Stan]
generated quantities {
  real log_lik[N];
  int y_rep[N];
  for (n in 1:N) {
    y_rep[n] = poisson_rng(lambda);
    log_lik[n] = poisson_lpmf(y[n] | lambda);
    }
}
\end{lstlisting}
 }
\end{frame}

\begin{frame}{PPC for count data -- Poisson model}
  
  \vspace{-1\baselineskip}
  \texttt{ppc\_dens\_overlay(y, yrep[1:50,])}

    \includegraphics[height=8cm]{poisson_ppc_dens_overlay.pdf}

\end{frame}

\begin{frame}{PPC for count data -- Poisson model}
  
  \vspace{-1\baselineskip}
  \texttt{ppc\_rootogram(y, yrep)}

    \includegraphics[height=8cm]{poisson_ppc_rootogram.pdf}

\end{frame}

\begin{frame}{PPC for count data -- Poisson model}
  
  \vspace{-1\baselineskip}
  \texttt{prop\_zero <- function(x) mean(x == 0)}\\
  \texttt{ppc\_stat(y, yrep, stat = "prop\_zero")}

    \includegraphics[height=7.5cm]{poisson_ppc_stat_propzero.pdf}

\end{frame}

\begin{frame}{PPC for count data -- hurdle truncated Poisson model}
  
  \vspace{-1\baselineskip}
  \texttt{ppc\_rootogram(y, yrep2)}

    \includegraphics[height=8cm]{poisson2_ppc_rootogram.pdf}

\end{frame}

\begin{frame}{PPC for count data -- hurdle truncated Poisson model}
  
  \vspace{-1\baselineskip}
  \texttt{prop\_zero <- function(x) mean(x == 0)}\\
  \texttt{ppc\_stat(y, yrep2, stat = "prop\_zero")}

    \includegraphics[height=7.5cm]{poisson2_ppc_stat_propzero.pdf}

\end{frame}

\begin{frame}{Further reading and examples}

  \begin{itemize}
  \item Gabry, Simpson, Vehtari, Betancourt, and Gelman
    (2019). Visualization in Bayesian
    workflow. \url{https://doi.org/10.1111/rssa.12378}.
  \item Graphical posterior predictive checks using the bayesplot package
    \url{http://mc-stan.org/bayesplot/articles/graphical-ppcs.html}
  \item Another demo \href{http://avehtari.github.io/BDA_R_demos/demos_rstan/ppc/poisson-ppc.html}{demos\_rstan/ppc/poisson-ppc.Rmd}
  \end{itemize}
  
\end{frame}

\begin{frame}{Sensitivity analysis}

  \begin{itemize}
  \item How much different choices in model structure and priors affect the results
    \begin{itemize}
      \item<2-> test different models and priors
      \item<3-> alternatively combine different models to one model
        \begin{itemize}
        \item e.g. hierarchical model instead of separate and pooled
        \item e.g. $t$ distribution contains Gaussian as a special case
      \end{itemize}
      \item<3-> robust models are good for testing sensitivity to ``outliers''
        \begin{itemize}
        \item e.g. $t$ instead of Gaussian
        \end{itemize}
    \end{itemize}
    \item<4-> Compare sensitivity of essential inference quantities
      \begin{itemize}
      \item extreme quantiles are more sensitive than means and medians
      \item extrapolation is more sensitive than interpolation
      \end{itemize}
    \end{itemize}

\end{frame}

\end{document}

%%% Local Variables: 
%%% TeX-PDF-mode: t
%%% TeX-master: t
%%% End: 
